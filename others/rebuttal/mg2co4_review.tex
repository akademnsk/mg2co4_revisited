	\documentclass[a4paper]{article}

\usepackage[T2A]{fontenc}
\usepackage[utf8]{inputenc}
\usepackage[russian,english]{babel}
\usepackage{amsmath,amsthm,amssymb}
\usepackage[affil-it]{authblk}
\usepackage{cite}
\usepackage{scrextend}
\usepackage{verbatim}
\usepackage{paralist}
\usepackage[mediumspace,mediumqspace,Grey,squaren]{SIunits}
\addtokomafont{labelinglabel}{\sffamily}
\usepackage{amsmath}
\usepackage{graphicx}
\usepackage[dvipsnames]{xcolor}
% \usepackage{xcolor}
 \usepackage{changepage} %for adjustwidth command (indenting paragraph}
\usepackage{SIunits}
\usepackage{miller}
\usepackage[version=3]{mhchem}
\usepackage[left=3cm,right=2cm]{geometry}
\usepackage{xr}
\externaldocument{na2co3_exp,na2co3_exp_supp}
\setlength{\parindent}{5ex}
\graphicspath{{figures/}}

\newcounter{reviewer}
\setcounter{reviewer}{0}
\newcounter{point}
\setcounter{point}{0}

\renewcommand{\thepoint}{P\,\thereviewer.\arabic{point}}
\newcommand{\reviewersection}{\stepcounter{reviewer} \bigskip \hrule
                  \section*{Reviewer \thereviewer}}
\newenvironment{point}
   {\refstepcounter{point} \bigskip \noindent {\textbf{Reviewer~Point~\thepoint} } ---\ }
   {\par }

\newcommand{\shortpoint}[1]{\refstepcounter{point}  \bigskip \noindent 
	{\textbf{Reviewer~Point~\thepoint} } ---~#1\par }

\newenvironment{reply}
   {\medskip \noindent \begin{sf}\textbf{Reply}:\  }
   {\medskip \end{sf}}

\newcommand{\shortreply}[2][]{\medskip \noindent \begin{sf}\textbf{Reply}:\  #2
	\ifthenelse{\equal{#1}{}}{}{ \hfill \footnotesize (#1)}%
	\medskip \end{sf}}

\begin{document}



\title{The formation of Mg-orthocarbonate through the reaction MgCO$_3$ + MgO = Mg$_2$CO$_4$ at Earth's lower mantle $P$--$T$ conditions \\ Manuscript ID: cg-2021-00140r} 
\author{Pavel N. Gavryushkin, Dinara N. Sagatova, Nursultan Sagatov, Konstantin D. Litasov}
\date{}
\maketitle


\section*{Response to the reviewers}
We would like to gratefully thank the reviewers for the careful read and precious suggestions, which make our manuscript more clear.  Changes have been done in accordance with all the suggestions. In the following we address the comments point by point.

\reviewersection

\subsection*{General considerations}
%%%


\begin{point}
All of the crystal structures models discussed in Figure 1 should be presented in parallel to Figure 2, and there cifs should be provided in SI. The quality of Fig 2 should be improved. In the current version we can not understand the connections of the polyhedrals directly.
\end{point}

	\begin{reply}
The figure of olivine-like Mg$_2$CO$_4$-Pnma was added to the Figure 2. 
As the other structures do not have stability fields we do not present their models, not to overload the figure.
The references on the works, based on which we have constructed structural models have been added.
The quality of the Fig2 have been improved and the figure illustrating connections between polyhedrons have been added.
As is is problematic to illustrate connections of polyhedrons for the whole structure, the interconnections of only separate polyhedrons have been shown.
	\end{reply}


\begin{point}
The previous exploration of Mg$_2$CO$_4$ should be introduced for a little bit more, like the paper Stability of Magnesite under the Lower Mantle Conditions, by Tomoo KATSURA, et al, published on Proc. Japan Acad., 67, Ser. B (1991) and its ref (14).
\end{point}

	\begin{reply}
The suggested reference and discussion have been added. 
	\end{reply}


\reviewersection

\begin{point}
	 The authors predicted that Mg$_2$CO$_4$-Pnma is isostructural to Mg$_2$SiO$_4$ (forsterite), not Ca$_2$CO$_4$-Pnma, please explain the origin from the structural packing perspective.  
\end{point}


\begin{reply}
The discussion of structural trends in the systems Mg$_2$SiO$_4$, Ca$_2$SiO$_4$ and Ca$_2$CO$_4$ have been added.
Mg$_2$CO$_4$-$P2_1/c$ is the deformed analogue of Ca$_2$CO$_4$-Pnma.
Mg$_2$CO$_4$-$Pnma$ is the lower pressure form, which is observed only for orthocarbonates with small cation like Mg and it is not observed for orthocarbonates of Ca, Sr, and Ba.
\end{reply}

\begin{point}
Electronic structure and chemical bonding is critical for understanding the phases and its connection with others, e.g. Mg$_2$SiO$_4$, Ca$_2$CO$_4$, which is missing in the manuscript.
\end{point}

\begin{reply}
The analysis of electronic denisty distribution (Mulliken charges, Baader analysis) of Mg$_2$CO$_4$-Pnma and isostructural to it Mg$_2$SiO$_4$-Pnma has been added.
\end{reply}

\begin{point}
The Raman spectra of predicted phases may be added for the later identification in experiments.
\end{point}

\begin{reply}
The Raman spectra of the predicted Mg$_2$CO$_4$-Pnma has been added.
\end{reply}


\reviewersection

\begin{point}
	This is an interesting theoretical paper on Mg orthocarbonate polymorphs. The only point that attracts my attention is why Mg$_2$CO$_4$ follows the olivine-larnite phase transition at high pressure (similar to Ca$_2$SiO$_4$) and not the Mg$_2$SiO$_4$ phase transitions that involve formation of such phases as poirierite, wadsleyite (probably also wadsleyite II) and, finally, ringwoodite (i.e. the spinel-structured phase). Were these phases among the candidates for the high-pressure theoretical phases? From Figure 1, it seems that the answer is 'yes'. But then why the authors refer in Figure 1 to Zn$_2$SiO$_4$ polymorphs and not to Mg$_2$SiO$_4$ polymorphs (ringwoodite, wadsleyite)? If they want to stress geological importance of their findings, it would be reasonable to indicate clearly parallels with the Mg$_2$SiO$_4$ polymophism, which is highly important for natural systems.
\end{point}

\begin{reply}
We thank the reviewer for this comment. 
The strucutures of ringwodite and wadsleyite are the same as some of the structures of Zn$_2$SiO$_4$ presented on Figure1. 
The results of the optimisation does not depend on what model are used for the preparation of Mg$_2$CO$_4$ structures, Zn$_2$SiO$_4$ or Mg$_2$SiO$_4$. 
In the revised manuscript we have changed Zn$_2$SiO$_4$ on Mg$_2$SiO$_4$ to show comparison of Mg$_2$CO$_4$ and Mg$_2$SiO$_4$ systems.
The enthalpies of Mg$_2$CO$_4$ in the form of poriereite and wadsleyite have been calculated and corresponding discussion has been added.


\end{reply}

\end{document}
